\section{Einleitung}

Nach seinen Enthüllungen der illegalen Aktivitäten der amerikanischen Geheimdienste im Jahr 2013 galt der Whistleblower Edward Snowden auch als eine Art Autorität für Datenschutz. So wurde darüber berichtet, dass Snowden in einem Guardian-Interview von einer Verwendung von Diensten wie Dropbox warnte, da diese die in der Cloud gespeicherten Daten lesen und auswerten können. Er empfahl, stattdessen einen \glqq{}Zero Knowledge\grqq{}-Anbieter wie Spideroak zu verwenden.\footnote{siehe \cite{snowden}} Damit meinte er eine konsequente Verschlüsselung und Entschlüsselung der Daten auf dem Gerät des Nutzers mit einem Schlüssel, auf den den der Cloud Provider keinen Zugriff besitzt. So kann der Cloudspeicheranbieter auch nicht auf die Daten zugreifen und diese analysieren - womit er \glqq{}nichts weiß\grqq{}.

Spätestens ab diesem Zeitpunkt richtete sich die öffentliche Aufmerksamkeit auf den Begriff \glqq{}Zero Knowledge\grqq{}. Spideroaks Verständnis desselben deckt sich jedoch nicht ganz mit der mathematischen Definition, oder genauer mit Zero-Knowledge-Beweisprotokollen, mit denen sich diese Arbeit beschäftigen wird. Dabei geht es um einen Beweiser, der die Lösung eines (mathematischen) Problems kennt, und einen Verifizierer, den dieser davon überzeugen soll, dass er im Besitz der Lösung ist. Der Verifizierer darf jedoch, unabhängig davon, was er versucht, kein zusätzliches Wissen zur Lösung des Problems aus der Konversation bekommen. 

Dies wurde auch dem \glqq{}Zero Knowledge\grqq{}-Anbieter Spideroak klar, sodass er um den Unterschied seines Produktes zur mathematischen Variante zu verdeutlichen am Anfang dieses Jahres auf seiner Website den Begriff zu \glqq{}No Knowledge\grqq{} änderte.\footnote{siehe \cite{noknowledge}}

Doch zurück zu den mathematischen Zero-Knowledge-Beweisen: Deren Konzept wurde erstmalig von Schafi Goldwasser, Silvio Micali und Charles Rackoff eingeführt. Die ersten Versionen der Arbeit darüber entstanden bereits im Jahr 1982. Jedoch wurde diese von drei größeren Konferenzen abgelehnt, bis sie schließlich im Jahr 1995 im Rahmen der \glqq{}Symposium on Theory of Computing\grqq{}, einer wissenschaftlichen Konferenz für theoretische Informatik, erstmalig publiziert wurde. Die Anwendung auf viele Probleme wurde von Oded Goldreich, Silvio Micali und Widgerson erschlossen, die zeigen konnten, dass sich für jede Sprache in NP ein Zero-Knowledge-Beweis konstruieren lässt.\footnote{aus \cite[Seite 3]{20yearszeroknowledge}}

In den folgenden Abschnitten werde ich Zero Knowledge formal definieren, zwei eher theoretische Beispiele präsentieren, erläutern, wie man für jede Sprache in NP einen Zero-Knowledge-Beweis konstruieren kann und anschließend verschiedene praktische Anwendungen von Zero Knowledge wie Zcash und ein Authentifizierungsprotokoll vorstellen.

\pagebreak
