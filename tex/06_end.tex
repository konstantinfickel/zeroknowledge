\section{Ausblick}

Eine etwas exotische Anwendungsmöglichkeit von Zero-Knowledge-Beweisen ist die Verifizierung von Atomsprengköpfen für Abrüstungsverhandlungen, wie sie in \cite{nuclear} beschrieben wird.

Um die Einhaltung von Verträgen zu überprüfen, bei denen die maximale Anzahl an Atomsprengköpfen beschränkt wird, wird ein Verfahren benötigt mit dessen Hilfe man testen kann, ob zwei Objekte wie beispielsweise angebliche Nuklearraketen tatsächlich gleich aufgebaut sind. Dabei dürfen jedoch die Inspektoren kein Wissen über den streng geheimen Aufbau erlangen. Genauer gesagt geht es um Messbilder, die durch das Bestrahlen der Sprengköpfe mit hochenergetischen Neutronen entstehen. Der naheliegende Ansatz wäre, ein spezielles Gerät zu konstruieren, welches die Bilder vergleicht und anschließend den Inspektoren nur mitteilt, ob die beiden Konstruktionen ähnlich zueinander sind oder nicht. Problematisch dabei ist, dass sich beide Parteien nicht vertrauen - und die überprüfte Atommacht befürchten muss, durch ein Hintertürchen ausspioniert zu werden; oder die Inspektoren sich nicht sicher sein können, ob das Gerät nicht manipuliert ist. Dies ist vor allem der Fall, da solche Geräte, nachdem sie mit dem klassifizierten Material in Kontakt gekommen sind, nicht weiter durch die Inspektoren überprüft werden dürfen, da sie noch geheime Informationen enthalten könnten.

Mit einem Zero-Knowledge-Beweis ließe sich jedoch dieses Problem lösen. Dabei werden verschiedene Sensorplatten zuerst mit Negativen der zu vergleichenden Testobjekte vorgeladen. Die Inspektoren dürfen dann entscheiden, mit welchen Platten sie welchen Atomsprengkopf messen. Danach wird mit der ausgewählten Platte die Neutronenabstrahlung des ausgewählten Testgeräts aufgenommen. Falls das Vorlageobjekt und das getestete Objekt identisch sind, sieht der Verifizierer nur ein Rauschen und kann mit hoher Wahrscheinlichkeit davon ausgehen, dass beide Objekte identisch sind. Hat die überprüfte Atommacht zwei nicht identische Objekte zum Vergleich angegeben, weichen Teile des Bildes vom normalen Rauschen ab. Dabei sind dann eventuell auch Teile des Aufbaus der Sprengköpfe erkennbar, was den Anreiz ehrlich zu sein erhöhen sollte. Nachdem der Test abgeschlossen ist, kann die Sensorplatte von den Inspektoren weiter überprüft werden, da sie keine geheimen Daten mehr enthält.

Wie hier ersichtlich, ist Idee von Zero-Knowledge-Beweisen also nicht nur für einfache mathematische Konstrukte zu gebrauchen, sondern kann auch in vielen praktischen Gebieten angewendet werden. Da das Gebiet noch relativ neu ist und jährlich neue Erfindung dazukommen, bleibt es spannend, in welchen Bereichen noch Zero-Knowledge-Beweise eingesetzt werden können und welche weiteren theoretischen Ergebnisse präsentiert werden.

\pagebreak
